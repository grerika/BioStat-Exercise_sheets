\chapter{Summary of the methods}
%\fancyhead[LO,RE]{\textsc{Summary of the methods}}

\begin{enumerate}
\item\label{both-cont}
 both are continuous (measured on the same subject):
 
 	\begin{enumerate}
	\item comparing the means of variables – the same thing about the same subjets, examining mean change: \textbf{paired t-test}
	\item  examining relationships between variables: \textbf{correlation, regression}
	\end{enumerate}
\item\label{one-cont} one continuous dependent variable divided into independent groups according to another, categorical variable: (that is, comparing means of groups)

	\begin{enumerate}
	\item number of groups=2: \textbf{two-sample t-test} (Independent t-test)
	\item number of groups>2: \textbf{One-way ANOVA} (Analysis of Variance)
	\end{enumerate}
\item both are categorical: examination of contingency tables, \textbf{$\chi^2$ test}
\end{enumerate}

\noindent In \ref{both-cont}. and \ref{one-cont}. we assumed that the samples come from normal distribution. If this assumption does not hold or we have ordered data, use \textbf{nonparametric methods based on ranks}.

\vfill
\section{Manual calculation}

\subsection{Effect of a new drug}
To test the effect of a new drug, the temperature was measured on the same 10 patients before and after the treatment. The mean of the differences is = 0.6, the standard deviation of the differences is SD=0.4062. Test whether the drug is effective or not; i.e., test whether the change in mean is different from 0 at $\alpha$=0.05 level.

\begin{enumerate}[a)]
	\item To test the effect of the drug, what is the appropriate test?\hrulefill

	Assumption(s)?\hrulefill

	\item 
	State the null and the alternative hypothesis
		
		H$_0$:	\hrulefill\\
		H$_A$:	\hrulefill

	\item Find the degrees of freedom: \hrulefill
		\quad
	standard error: \hrulefill


	test statistic: \hrulefill
		\quad
	critical value: \hrulefill

	\item Decide whether the difference is significant or not: \hrulefill


		State a conclusion: \hrulefill

	\item \emph{Alternative solution using confidence interval}

	Find the 95\% confidence interval: \hrulefill

	Decide whether the difference is significant or not: \hrulefill

	Conclusion: \hrulefill
\end{enumerate}


\subsection{Drug side effect}
Two medicines are being compared regarding a particular side effect; 100 similar patients are split randomly into two groups, one of each group. In the group of drug A, 10 side effects were observed while in the group of drug B only 2 side effects were observed. Test whether drug and side effects are independent!\bigskip

	\begin{tabular}{c|cc|c}
	\toprule
		&\multicolumn{2}{c|}{\textbf{SIDE EFFECTS}}\\
	\textbf{DRUG}	&\textbf{yes}	&\textbf{no}	&{\color{white} Total}\\
	\midrule
	\textbf{A}	&	&	&\\
	\textbf{B}	&	&	&\\
	\midrule
				&		&\\
	\bottomrule
	\end{tabular}
\begin{enumerate}[a)]
	\item Give the percentage of side effects in 
	
		group A: \hrulefill\quad group B:  \hrulefill
	\item What is the appropriate test? 
	
		\hrulefill 
		

			Assumption(s) of the test:
		
		\hrulefill
	\item State the null and the alternative hypothesis
	
	H$_0$: 	\hrulefill 	\\
	H$_A$: \hrulefill 	
	
\item
	Find the test statistic: \hrulefill
	
\item
	Find the degrees of freedom: 	\hrulefill \quad Critical value ($\alpha=0.05$): \hrulefill 	


	\item Decide whether the difference is significant or not:\\
	
	 \hrulefill 	
	
	\item
	State a conclusion:
	
	 \hrulefill 	
	
	
	\end{enumerate}
	
	
\subsection{Association}
Based on 11 pairs of data, the coefficient of correlation is r=0.8. Is the correlation significant at 5\% level? What is your opinion about the direction and about the strength of the association? 	


\begin{enumerate}[a)]
	\item State the null and the alternative hypothesis
	
	H$_0$: 	\hrulefill 	\\
	H$_A$: \hrulefill 	
	
	\item Find the test statistic: \hrulefill
	
	\item Find the degrees of freedom: 	\hrulefill \quad Critical value: \hrulefill 	


	\item		Decision
	
	 \hrulefill 	
	
	\item
Interpretation
	
	 \hrulefill 	
	
	
	\end{enumerate}
	
	
	\vfill\clearpage
	 	
\section{Calculation using R}

Open the file \data{quest2016.csv}. This file contains data of first year medical students.
\subsection{Examine the relationship between ideal body mass and body mass}
Examine the relationship between the body mass (\variable{mass}) and the ideal body mass (\variable{ideal\_mass})! Let the present mass be the independent variable. Is there a linear relationship between body mass at present and ideal body mass?


\begin{enumerate}[a)]
	\item The name of the appropriate test:	
				\hrulefill 
				
		Assumption(s) of the test: \hrulefill
				
	\item	State the null and the alternative hypothesis
			
			H$_0$: 	\hrulefill 	\\
			H$_A$: \hrulefill 	
	\item
	
		$r=$  \rule{15mm}{.4pt}	\quad it's meaning: \hrulefill

		$p$-value:\hrulefill 	
		
	\item Significance (decision): \hrulefill
				
		Explain your result in the context of the problem: \hrulefill
	
		 \hrulefill
						
	\item Equation of the line: \hrulefill 	
		
		
		
			Based on the equation, find the ideal body mass to a 60~kg  (actual) mass: \hrulefill 
					
			 \hrulefill
				
 \end{enumerate}
 
 
 \subsection{Mean change of mass}
 Students were asked about they body mass at present (variable \variable{mass}) and body weight 3 years ago (\variable{mass3}). Find whether the mean change of mass is significant or not at 5\% level.
 
 
 \begin{enumerate}[a)]
 			\item The name of the appropriate test:	
				\hrulefill 
				
			\item
					Assumption(s) of the test: \hrulefill
				
				\hrulefill
			\item		
			State the null and the alternative hypothesis
			
			H$_0$: 	\hrulefill 	\\
			H$_A$: \hrulefill 	
			\item \emph{Descriptive statistics}
			
			
				\begin{large}
					\begin{center}
						\begin{tabular}{l||l|l|l}
						\toprule
						variable		& mean	& standard deviation & sample size\\
						\midrule
						mass		&&&\\
									&&&\\
						mass3	&&&\\
						\bottomrule
						\end{tabular}
					\end{center}						
				\end{large}

			\item \emph{Result of the test}
				\\
				
				test statistic: \rule{30mm}{.4pt}	degrees of freedom:  \rule{30mm}{.4pt}	$p$ -value: \hrulefill
				\\

					Decision: 	\hrulefill
		
			\item Explain your result in the context of the problem:
			
			 \hrulefill

\end{enumerate}




 \subsection{Age of boys and girls}
Compare the mean age of boys and girls! (variables \variable{age}, \variable{gender})
 
 \begin{enumerate}[a)]
 			\item The name of the appropriate test:	
				\hrulefill 
				
			\item
					Assumption(s) of the test: \hrulefill
				
				\hrulefill
			\item		
			State the null and the alternative hypothesis
			
			H$_0$: 	\hrulefill 	\\
			H$_A$: \hrulefill 	
			\item \emph{Descriptive statistics}
			
			
				\begin{large}
					\begin{center}
						\begin{tabular}{l||l|l|l}
						\toprule
								& mean	& standard deviation & sample size\\
						\midrule
						boy	&&&\\
									&&&\\
						girl	&&&\\
						\bottomrule
						\end{tabular}
					\end{center}						
				\end{large}

			\item \emph{Result of the test}
				\\
				
				test statistic: \rule{30mm}{.4pt}	degrees of freedom:  \rule{30mm}{.4pt}	$p$ -value: \hrulefill
				\\

					Decision: 	\hrulefill
		
			\item Explain your result in the context of the problem:
			
			 \hrulefill		
			 
			 \item (Result of the test of equality of variances: \hrulefill)	
	
\end{enumerate}



\subsection{Eye color and gender}
Examine the relationship between the answers of eye color and gender. Is the gender and eye color of students independent (variable \variable{gender} and \variable{eye})?

\begin{enumerate}[a)]
	\item What is the appropriate test? 	\hrulefill
	
	\item		
			State the null and the alternative hypothesis
			
			H$_0$: 	\hrulefill 	\\
			H$_A$: \hrulefill 	


	\item \emph{Descriptive statistics: 	}
	\vspace{5em}

	\item What is the assumption of the test? \hrulefill
	
	 	Does it come true? \hrulefill
	 	
	 \item Find the degrees of freedom: \hrulefill\quad
		 the test statistic: \hrulefill\quad 
		 critical value: \hrulefill\quad 
		 p-value: 	\hrulefill\quad
	\item State a conclusion: \hrulefill
	
\end{enumerate}
