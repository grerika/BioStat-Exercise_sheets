\chapter[Further problems]{Type of problems\\ not practiced during the semester}

\section{Fourfold (2$\times$2) tables}
\begin{enumerate}
\item In a study of measurement of agreement, the observed and expected probabilities were 0.85 and 0.5, respectively. Calculate the kappa statistic! 
\item In a study 40 HPV positive tests of 50 abnormal cervical samples and 10 HPV positive tests of 60 normal cervical samples were detected. Calculate the odds ratio! 
\item The risk of HPV infection for smokers was measured in a study. The calculated odds ratio was 1.58 with 95\% CI [1.061 - 2.398]. We decide... 
\item The risk of HPV infection for smokers was measured in a study. The calculated odds ratios was 1.58 with 95\% CI [0.961 - 2.598]. We decide ... 
\item In a study 8 HPV positive tests of 20 abnormal cervical samples and 10 HPV positive tests of 20 normal cervical samples were detected. Calculate the odds ratio! 
\end{enumerate}

\section{Nonparametric tests}
\begin{enumerate}
\item Find the ranks to the following data: 199, 126, 81, 68, 112, 112.

\end{enumerate}

\section{Survival analysis}
\begin{enumerate}
\item In a study the average period of time of free of disease was 3.1 year and SE=0.44. Compare this result to the reference 2.2 years of survival rate using a 95\% confidence interval!
\item In a study the average period of time of free of disease was 3.1 year and SE=0.44. Compare this result to the reference 2.2 years of survival rate! What is the null hypothesis ? 
\item In a cohort study the first year the interval survival rate was 0.99. In the following four years the annual interval survival rates were 0.98, 0.97, 0.96 and 0.95, respectively. Calculate the 5-year cumulative survival rate!
\item In a cohort study the first year the interval survival rate was 0.90. In the following four years the annual interval survival rates were 0.90, 0.90, 0.90 and 0.90, respectively. Calculate the 5-year cumulative survival rate!
\end{enumerate}
