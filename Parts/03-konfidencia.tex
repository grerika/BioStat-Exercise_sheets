\chapter[Confidence intervals]{Confidence\\ intervals}
\section[Confidence interval for the population mean if the population SD is known]{
Confidence interval for the population mean ($\mu$) if the population standard deviation ($\sigma$) is known}

\subsection{Assume that the heights of first year pharmaceutical students are normally distributed with a mean of $\mu = 175$ and a standard deviation of $\sigma=10$.}

\begin{enumerate}[a)]
\item What percentage of heights are above 175 cm? 	\hrulefill
\item What percentage of heights are below 175 cm? 	\hrulefill
\item What percentage of heights are between 155 and 195? 	\hrulefill
\item What percentage of heights are below 155 cm? 	 \hrulefill
\item Calculate the mean and the standard error of the mean of a sample of 36 cases derived from this population.

	\begin{enumerate}[i)]
	\item mean? \hrulefill
	\item standard error? \hrulefill
	\end{enumerate}

\item The mean of another random sample with 36 number of cases is 172. 
	\begin{enumerate}[i)]
	\item Calculate the 95\% confidence interval.  \hrulefill
	\item What is the meaning of the 95\% CI? \hrulefill
	\item Compare the population mean with the 95\% CI calculated. 
		Is the population mean included in the 95\% CI? \hrulefill
	\end{enumerate}
\end{enumerate}


\section[Confidence interval for the population mean if the population SD is unknown]{
Confidence interval for the population mean ($\mu$)
if the population standard deviation ($\sigma$) is unknown
}
	
	
\subsection{(Example from Altman). In a trial we actually observed a mean serum albumin of 34.46 g/l with a standard deviation of 5.834 g/l from a sample of 21 patients with primary biliary cirrhosis. }


%SE (standard error): \hrulefill
\begin{enumerate}[a)]
\item Find the 95\% confidence interval.

	\begin{description}
	\item[$\alpha:$] \hrulefill\quad n: 	\hrulefill
	\item[mean:] \hrulefill\quad \textbf{SE:} \hrulefill
	\item[degrees of freedom:]  \hrulefill\quad \textbf{$t_\alpha$:} \hrulefill
	\item[mean - $t_\alpha$ SE:] \hrulefill \quad\quad \textbf{mean + $t_\alpha$ SE:} \hrulefill
	\item[Confidence interval:] \hrulefill
		
		\textit{Meaning:} $\P(\hspace{3em}< \textrm{ true population mean } < \hspace{3em}) = 0.95$

		We can be 95\% confident from this study that the true mean serum albumin among all such patients lies somewhere in the range \rule{15mm}{.4pt} to \rule{15mm}{.4pt} g/l, with 34.46 as our best estimate. This interpretation depends on the assumption that the sample of 21 patients is representative of all patients with the disease.
	\end{description}

\clearpage
\item Find the 99\% confidence interval and compare to the 95\% CI.
	\begin{description}
	\item[$\alpha:$] \hrulefill\quad n: 	\hrulefill
	\item[mean:] \hrulefill\quad \textbf{SE:} \hrulefill
	\item[degrees of freedom:]  \hrulefill\quad \textbf{$t_\alpha$:} \hrulefill
	\item[mean - $t_\alpha$ SE:] \hrulefill \quad\quad \textbf{mean + $t_\alpha$ SE:} \hrulefill
	\item[Confidence interval:] \hrulefill
		
		Meaning: $\P(\hspace{3em}< \textrm{ true population mean } < \hspace{3em}) = 0.99$
	\end{description}

\item Suppose that the above data were observed from a sample of 210 patients. Find the 95% confidence interval.
	\begin{description}
	\item[$\alpha:$] \hrulefill\quad n: 	\hrulefill
	\item[mean:] \hrulefill\quad \textbf{SE:} \hrulefill
	\item[degrees of freedom:]  \hrulefill\quad \textbf{$t_\alpha$:} \hrulefill
	\item[mean - $t_\alpha$ SE:] \hrulefill \quad\quad \textbf{mean + $t_\alpha$ SE:} \hrulefill
	\item[Confidence interval:] \hrulefill
	\end{description}		
\end{enumerate}

	


	
\subsection{In a study, systolic blood pressure of 10 healthy women was measured}
Calculate the 95\% confidence interval for the population mean,  if the 

\begin{itemize}
 \item  mean was 119, the standard error was 0.664
 
  \hrulefill
 \item  mean was 119, the standard error was 2.1
 
 	\hrulefill
\end{itemize}
Compare the length of this confidence intervals!

\noindent \hrulefill
\subsection{Questions}

	
\begin{enumerate}[a)]
\item Which is wider, a 95\% or a 99\% confidence interval? 	

 \hrulefill
\item   When you construct a 95\% confidence interval, what are you 95\% confident about? 	

	 \hrulefill
\item When computing a confidence interval, when do you use $t$ and when do you use $u$? 	

\hrulefill
\item What does the confidence interval width depend on? 	

\hrulefill
\end{enumerate}

\clearpage

\section{Calculations with R}

\subsection{Calculate the following statistics for \variable{mass} using the \quest!}

\begin{description}
\item[Sample size: ] \hrulefill \quad \textbf{Mean:} \hrulefill
\item[Standard deviation: ] \hrulefill \quad \textbf{Standard error:} \hrulefill
\end{description}

\begin{enumerate}[a)]
\item Assuming that the variable mass follows a normal distribution, determine the 
	\begin{enumerate}[i)]
	\item 95\% confidence interval: \hrulefill
	\item 99\% confidence interval: \hrulefill 	
	\end{enumerate}
\item Can we state that the mean body mass in the population of students is 70? 	

  \hrulefill	

	Explain. \hrulefill
\end{enumerate}


\subsection{Calculate the following statistics for \variable{height} using the \quest!}

\begin{description}
\item[Sample size: ] \hrulefill \quad \textbf{Mean:} \hrulefill
\item[Standard deviation: ] \hrulefill \quad \textbf{Standard error:} \hrulefill
\end{description}

\begin{enumerate}[a)]
\item Assuming that the variable height follows a normal distribution, determine the 
	\begin{enumerate}[i)]
	\item 95\% confidence interval: \hrulefill
	\item 99\% confidence interval: \hrulefill 	
	\end{enumerate}
\item Can we state that the mean height in the population of students is 170? 	

  \hrulefill	

	Explain. \hrulefill
\end{enumerate}

\section{Homework}

\subsection{A researcher set a 95\% confidence interval on the mean length of fish in a recreational lake and found it to be from 6.2 to 8.7 inches.} 
\textbf{Which of the following is a proper interpretation of this interval?}

\begin{enumerate}[(A)]
\item Of the fish in the recreational lake, 95\% are between 6.2 and 8.7 inches long.
\item We are 95\% confident that the sample mean length of fish in the recreational lake is between 6.2 and 8.7 inches.
\item We are 95\% confident that the population mean length of fish in the recreational lake is between 6.2 and 8.7 inches.
\item There is a 95\% chance that a randomly selected fish from the recreational lake will be between 6.2 and 8.7 inches.
\end{enumerate}

\subsection{An industrial designer wants to determine the average amount of time it takes an adult to assemble an ''easy to assemble'' toy. A sample of 16 times yielded an average time of 19.92 minutes, with a sample standard deviation of 5.73 minutes. Assuming normality of assembly times, provide a 95\% confidence interval for the mean assembly time.}
	
