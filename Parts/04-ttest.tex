\chapter[One-sample, two-sample and paired $t$-tests]{One-sample, two-sample\\ and paired $t$-tests}
\section{One-sample $t$-test for the mean of a normal population}
\subsection{The following are the systolic blood pressures (mm Hg) of $n = 9$ patients undergoing drug therapy for hypertension:}

\begin{center}
182.00  152.00  178.00  157.00  194.00  163.00  144.00  114.00  174.00
\end{center}


\textbf{The mean = 162 mm Hg, the standard deviation SD = 23.92.}


\begin{enumerate}[a)]
\item Find the standard error.	\hrulefill
\item Find the 95\% confidence interval for the population mean.	
	 \hrulefill

	What is the meaning of this interval? 	  \hrulefill	

\item \textbf{We would like to test whether the sample is drawn from a population where $\mu = 130$.}

Find the null- and alternative hypothesis.
	
	H$_0$:	\hrulefill
	
	H$_A$:	\hrulefill
	
	\begin{description} %[i)]
	\item[Based on the confidence interval]
	Can we conclude with 95\% confidence on the basis of these data that the population mean is different from 130? 
	Explain your decision. 
	
	\hrulefill
	
	\item[Based on the $t$ test-statistics]
	


	Is the population mean significantly different from 130 at 5\% level?
	
	$t = \frac{\textrm{mean} - 130}{SE} =$ 	\hrulefill
	
Compare its absolute value to the t-value of the table. 	 \hrulefill
		
		Explain your decision. 	\hrulefill
	\item[Based on the $p$-value]
	The $p$-value given by R is $p = 0.004$.
	
	Is there a significant difference from the hypothesized population mean 130 at 5\% level?

		\hrulefill

	\end{description}
\item \textbf{We would like to test whether the sample is drawn from a population where $\mu = 150$.}
	
Find the null- and alternative hypothesis.
		
		H$_0$:	\hrulefill
		
		H$_A$:	\hrulefill
		
		\begin{description}
		\item[Based on the confidence interval]
	Can we conclude with 95\% confidence on the basis of these data that the population mean is different from 150? 
	Explain your decision. 
	
	\hrulefill
	
	\item[Based on the $t$ test-statistics]
	


	Is the population mean significantly different from 150 at 5\% level?
	
	$t = \frac{\textrm{mean} - 150}{SE} =$ 	\hrulefill
	
Compare its absolute value to the t-value of the table. 	 \hrulefill
		
		Explain your decision. 	\hrulefill
	\item[Based on the $p$-value]
			The $p$-value given by R is $p = 0.171$.
	
			Is there a significant difference from the hypothesized population mean 150 at 5\% level?
		
				\hrulefill	
		\end{description}
\end{enumerate}
	
\section{Paired $t$-test}
\subsection{The effect of saline on the blood PH was examined in a certain disease. The blood PH value was measured two times: before the treatment and 20 minutes later, after infusion of saline ($n = 18$). Is there a significant change in mean blood PH at 5\% level? }

	\begin{minipage}{0.2\textwidth}
	\flushright
		\begin{tabular}{cc}
		\toprule
		0’&20’\\
		\midrule
		7.43&7.43\\
		7.39&7.39\\
		7.37&7.38\\
		7.43&7.42\\
		7.39&7.39\\
		7.36&7.41\\
		7.38&7.38\\
		7.39&7.39\\
		7.34&7.41\\
		7.32&7.35\\
		7.40&7.39\\
		7.32&7.33\\
		7.42&7.39\\
		7.42&7.4\\
		7.37&7.36\\
		7.37&7.39\\
		7.39&7.37\\
		7.43&7.48\\
		\bottomrule
		\end{tabular}
	\end{minipage}
	\begin{minipage}{0.7\textwidth}
		\centering
		\begin{tabular}{cccc}
		\toprule
		& \multicolumn{3}{c}{\textbf{Descriptive statistics}}	\\
		
				& 0’		& 20’ 		& difference\\
			\midrule
		mean 	& 7.3844	& 7.3922	& -0.00778\\
		SD		& 0.03485	& 0.03264	& 0.02691\\
		\bottomrule
		\end{tabular}\bigskip		
		
		mean-SD diagram: 
		
		\vspace{12em}
	\end{minipage}	
		
\begin{enumerate}[a)]
\item The name of the appropriate test: \hrulefill
\item H$_0$:	 \hrulefill


	 H$_A$:	 \hrulefill
\item $t =$ 	 \hrulefill\quad  df = \hrulefill \quad critical $t$-value ($t_\alpha$)= \hrulefill
\item Decision: 	 \hrulefill

	Conclusion: \hrulefill


\item Check your calculation using results of R.


\lstinputlisting[float=h,frame=tblr]{Code/05-ttest1.txt}


	\begin{enumerate}[i)]
	\item Find the 95\% confidence interval for the difference. 	 \hrulefill
	
	
		Decision based on the confidence interval: 		\hrulefill
	\item $t =$ 	 \hrulefill\quad df = 	\hrulefill	
	
		 Decision based on $t$-value: 	\hrulefill
	\item $p$-value = \hrulefill 
	
		Decision based on $p$-value: 	\hrulefill
	\end{enumerate}
\end{enumerate}


\subsection{The effect of Na-lactate on the blood PH was examined in a certain disease. The blood PH value was measured two times: before the treatment and 20 minutes later, after infusion of Na-lactate (n = 20). Is there a significant change in mean blood PH at 5\% level?}

	\begin{minipage}{0.2\textwidth}
	\flushright\small
		\begin{tabular}{cc}
		\toprule
		0’&20’\\
		\midrule
			7.42&7.46\\
			7.36&7.43\\
			7.4 &7.46\\
			7.43&7.48\\
			7.38&7.42\\
			7.32&7.45\\
			7.37&7.46\\
			7.36&7.48\\
			7.34&7.45\\
			7.31&7.37\\
			7.34&7.47\\
			7.37&7.43\\
			7.42&7.48\\
			7.42&7.43\\
			7.46&7.51\\
			7.37&7.41\\
			7.45&7.48\\
			7.42&7.44\\
			7.42&7.37\\
			7.41&7.45\\
		\bottomrule
		\end{tabular}
	\end{minipage}
	\begin{minipage}{0.7\textwidth}
		\centering
		\begin{tabular}{cccc}
		\toprule
		& \multicolumn{3}{c}{\textbf{Descriptive statistics}}	\\
		
				& 0’		& 20’ 		& difference\\
			\midrule
		mean 	& 7.3885	& 7.4465	& -0.058\\
		SD		& 	0.04258	& 	0.03573	& 0.04336\\
		\bottomrule
		\end{tabular}\bigskip		
		
		mean-SD diagram:
		
		\vspace{12em}
	\end{minipage}	
		
				
\begin{enumerate}[a)]
\item The name of the appropriate test: \hrulefill
\item H$_0$:	 \hrulefill


	 H$_A$:	 \hrulefill
\item $t =$ 	 \hrulefill\quad  df = \hrulefill \quad critical $t$-value ($t_\alpha$)= \hrulefill
\item Decision: 	 \hrulefill

	 Conclusion: \hrulefill


\item Check your calculation using results of R.


\lstinputlisting[float=h,frame=tblr]{Code/05-ttest2.txt}


	\begin{enumerate}[i)]
	\item Find the 95\% confidence interval for the difference. 	 \hrulefill
	
	
		Decision based on the confidence interval: 		\hrulefill
	\item $t =$ 	 \hrulefill\quad df = 	\hrulefill	
	
		 Decision based on $t$-value: 	\hrulefill
	\item $p$-value = \hrulefill 
	
		Decision based on $p$-value: 	\hrulefill
	\end{enumerate}
\end{enumerate}






\section{Two-sample $t$-test}
\subsection[Back pain comparison]{In a medical study\footnote{Manchikanti L, Cash K, McManus C, Pampati V and Benyamin R. Randomized, Double-Blind, Active-Controlled Trial of Fluoroscopic Lumbar Interlaminar Epidural Injections in Chronic Axial or Discogenic Low Back Pain: Results of 2-Year Follow-Up. Pain Physician 2013; 16:E491-E504  ISSN 2150-1149.} two treatments were applied to back pain on 60-60 randomly assigned patients.
The following table contains demographic and clinical characteristics.  The continuous variables were compared using two-sample $t$-test. Check the $p$-values and the assumptions of the hypothesis test using the \data{independent\_ttest\_n\_mean\_sd.R} file!}
	
		\begin{center}
		\includegraphics[trim={1.5cm 15.3cm 1.5cm 3.5cm},clip,width=0.8\textwidth]{Plots/Statisztika-ketmintas-tablak.pdf}
		\end{center}
		
	
	\subsubsection{Age variable}
	
	\begin{enumerate}[a)]
	\item The name of the appropriate test:	\hrulefill
	
		Why we can't apply paired $t$-test? \hrulefill
	\item H$_0$:	\hrulefill

		 H$_A$:	\hrulefill
	\item Assumptions of the test:	\hrulefill
			\\
			Normality fulfillment: \hrulefill
			\\
			Equal variances fulfillment: \hrulefill
			
	\item \emph{Descriptive statistics}: 
		%
		\hrulefill
		
	\item Is there a significant difference?\hrulefill\quad Why? \hrulefill
	
	\item Check the $p$-value with R. Was the chosen method appropriate? \hrulefill
	\end{enumerate}
	
	
		\subsubsection{Duration of Pain  variable}
		
		\begin{enumerate}[a)]
		\item The name of the appropriate test:	\hrulefill
		
			Why we can't apply paired $t$-test? \hrulefill
		\item H$_0$:	\hrulefill
	
			 H$_A$:	\hrulefill
		\item Assumptions of the test:	\hrulefill
				\\
				Normality fulfillment: \hrulefill
				\\
				Equal variances fulfillment: \hrulefill
				
		\item \emph{Descriptive statistics}: 
		%
			\hrulefill
			
		\item Is there a significant difference?\hrulefill\quad Why? \hrulefill
		
		\item Check the $p$-value with R. Was the chosen method appropriate? \hrulefill
		\end{enumerate}

\clearpage
\subsection{Compare beers}
Using \data{BEER.csv} datafile,  compare the calorie content and prices of \emph{LIGHT} and \emph{NONLIGHT} american beers  (variables \variable{LIGHT} and \variable{CALORIES})!
The population means of light and nonlight beers are the same?

\subsubsection{Calories}
	\begin{enumerate}[a)]

	\begin{multicols}{2}
	\item Applied test:
	
	\hrulefill
	\item H$_0$: \hrulefill\quad H$_A$: \hrulefill
	
		
	\item Assumption(s) of the test: \hrulefill

		Fulfillment(s)? \hrulefill 
		
		 \hrulefill

		\item Descriptive statistics
		\begin{center}\small
			\begin{tabular}{r|C{2cm}|C{2cm}}
			\toprule
						& light & nonlight\\
			\midrule
			sample size	&&\\
			mean		&&\\
			SD		&&\\
			SE			&&\\
			\bottomrule
			\end{tabular}
		\end{center}
	\end{multicols}

	\item \emph{Result}
	
		Test-statistics: \hrulefill\quad df: \hrulefill\quad $p$-value: \hrulefill
		
		Is there a significant difference?? \rule{5em}{0.4pt} Why? \hrulefill
	\end{enumerate}
	
\subsubsection{Prices}
	\begin{enumerate}[a)]
		\begin{multicols}{2}
	


	\item Applied test:
	
	\hrulefill
	\item H$_0$: \hrulefill\quad H$_A$: \hrulefill
	
		
	\item Assumption(s) of the test: \hrulefill

		Fulfillment(s)? \hrulefill 
		
		 \hrulefill

		\item Descriptive statistics
		\begin{center}\small
			\begin{tabular}{r|C{2cm}|C{2cm}}
			\toprule
						& light & nonlight\\
			\midrule
			sample size	&&\\
			mean		&&\\
			SD		&&\\
			SE			&&\\
			\bottomrule
			\end{tabular}
		\end{center}
	\end{multicols}
	\item \emph{Result}
	
		Test-statistics: \hrulefill\quad df: \hrulefill\quad $p$-value: \hrulefill
		
		Is there a significant difference?? \rule{5em}{0.4pt} Why? \hrulefill
	\end{enumerate}
	
\section{Calculations with R}

\subsection{Open the file \data{befafter.csv}. A study was conducted to determine weight loss, body composition, etc. in obese women before and after 12 weeks of treatment with a very-low-calorie diet. Column \variable{BEFORE} and \variable{AFTER} contain weights of 9 women. We wish to know if these data provide sufficient evidence to allow us to conclude that the treatment is effective in causing weight reduction in obese women. Let $\alpha= 0.05$}

		
\begin{enumerate}[a)]

\item The name of the appropriate test:	 \hrulefill

	H$_0$:	 \hrulefill\\
		 H$_A$:	 \hrulefill


\begin{multicols}{2}
		\begin{center}\small
		\begin{tabular}{lC{2cm}C{2cm}}
			\toprule	
				\textit{}		& mean & SD\\
			\midrule
			Before &&\\
			After  &&\\
			difference &&\\
			\bottomrule
		\end{tabular}\medskip
	\end{center}

	 95\% CI for the difference:  \hrulefill
	%

	 $t =$ 	 \hrulefill\quad degrees of freedom = 	 \hrulefill\quad $t_\alpha$=	\hrulefill	
	 
	 $p$-value = \hrulefill 
	

\end{multicols}
	
	\item	Decision: \hrulefill

\end{enumerate}


	
\subsection{Open the file \quest!\\ Compare the mean body mass of boys and girls (at 5\% level).}
	

	\begin{enumerate}[a)]

	\item The name of the appropriate test:\hrulefill
	\item H$_0$:	\hrulefill
	
		H$_A$:	\hrulefill
		
	\item Assumptions: 	\hrulefill
	\end{enumerate}
	
		\begin{multicols}{2}
		\begin{center}
			\begin{tabular}{r|C{2cm}|C{2cm}}
			\toprule
						& boys & girls\\
			\midrule
			sample size	&&\\
			mean		&&\\
			SD		&&\\
			SE			&&\\
			\bottomrule
			\end{tabular}
		\end{center}

\subsubsection*{Equality of variances $\alpha=5\%$}
		\begin{enumerate}[a)]
		\item $p$-value: \hrulefill	
		\item Decision about the equality of variances:	
		
		\hrulefill
		\end{enumerate}
		\end{multicols}
\subsubsection*{Equality of population means $\alpha=5\%$}
		\begin{enumerate}[a)]
		\item $t=$ \hrulefill \quad df?	\hrulefill\quad			$p=$ \hrulefill
		\item 95\% CI of the difference: \hrulefill
		\item Decision about the equality of population means: \hrulefill

			Conclusion: \hrulefill		
		\end{enumerate}



\section{Homework}
%	
%\subsection{The systolic blood pressure of 6 patients was measured before and after a new drug. The mean of the sample differences is 6 mmHg, the standard error of the differences is 4.65. Is there a significant change in blood pressure at 5\% and at 1\% level?}
%
%	\begin{minipage}{0.45\textwidth}
%	$\alpha=5\%$
%	
%		\begin{enumerate}[a)]
%		\item Appropriate test: \hrulefill
%		\item H$_0:$	 \hrulefill
%
%			 H$_A$:	 \hrulefill
%		\item t: 	 \hrulefill
%
%			degrees of freedom: \hrulefill 
%
%			critical value: \hrulefill
%		\item Decision: 	 \hrulefill
%
%				Conclusion: \hrulefill
%		\end{enumerate}
%	\end{minipage}
%	\hfill
%	\begin{minipage}{0.45\textwidth}
%		$\alpha=1\%$
%		
%		\begin{enumerate}[a)]
%		\item Appropriate test: \hrulefill
%		\item H$_0:$	 \hrulefill
%
%			 H$_A$:	 \hrulefill
%		\item t: 	 \hrulefill
%
%			degrees of freedom: \hrulefill 
%
%			critical value: \hrulefill
%		\item Decision: 	 \hrulefill
%
%				Conclusion: \hrulefill
%		\end{enumerate}
%	\end{minipage}

\subsection{The body mass of 16 patients was measured before and after a special diet. The mean of the sample differences is 5 kg, the standard deviation of the differences is 2.5. Is there a significant change in body mass at 5\% and at 1\% level?}
	
	\begin{multicols}{2}
	
	$\alpha=5\%$
		\begin{enumerate}[a)]
		\item Appropriate test: \hrulefill
		
				H$_0:$	 \hrulefill
				H$_A$:	 \hrulefill
	\item t: 	 \hrulefill

			degrees of freedom: \hrulefill  \quad 
			critical value: \hrulefill
		\item Decision: 	 \hrulefill

				Conclusion: \hrulefill
		\end{enumerate}
	\columnbreak
			$\alpha=1\%$
		\begin{enumerate}[a)]
		\item Appropriate test: \hrulefill

			H$_0:$	 \hrulefill
			 H$_A$:	 \hrulefill
		\item t: 	 \hrulefill

			degrees of freedom: \hrulefill  \quad 
			critical value: \hrulefill

		\item Decision: 	 \hrulefill

				Conclusion: \hrulefill
		\end{enumerate}
	\end{multicols}	

\subsection[LWTBWT.csv]{Open the file \data{LWTBWT.csv} and compare the mean body weight of newborn babies (variable \variable{BWT}) by smoking habits of the mother (variable  \variable{SMOKE} 0 no, 1 yes)} 


\subsection[ANTHROPOMETRICS.csv]{Open the file \data{ANTHROPOMETRICS.csv} and compare the body height of boys and girls. Find other variables to be compared and find the appropriate test.}

\subsection[CALC.csv]{Open the file \data{CALC.csv}! Here systolic blood pressures are given before and after a calcium treatment in two groups. Find problems where paired $t$-tests can be used. Find problems where two-sample $t$-tests can be used. }

\subsection[NEWDRUG.csv]{Open the file \data{NEWDRUG.csv} and find problems where paired $t$-tests can be used. Find problems where two-sample $t$-tests can be used.}
