\documentclass[border=2mm]{standalone}
\usepackage{pgfplots}

\begin{document}

\pgfmathdeclarefunction{gauss}{3}{%
  \pgfmathparse{1/(#3*sqrt(2*pi))*exp(-((#1-#2)^2)/(2*#3^2))}%
}

\begin{tikzpicture}
\begin{axis}[
%  no markers, 
  domain=-3:9, 
  samples=200,
  ymin=0,
  ymax=0.6,
  axis lines*=left, 
%  xlabel=$x$,
  every axis y label/.style={at=(current axis.above origin),anchor=south},
  every axis x label/.style={at=(current axis.right of origin),anchor=west},
 % height=5cm, 
 % width=12cm,
%  xtick=\empty, 
  %ytick=\empty,
  %enlargelimits=false, 
  clip=false, 
  axis on top,
  grid=both,
  xtick = {-3,-2,-1,0,1,2,3,4,5,6,7,8,9},
  %xticklabels = {,,-1,,,,3,,,7,,},
  yticklabels={0,0.05,0.10,0.15,0.20,0.25,0.30,0.40,0.50,0.60},  
  ytick = {0,0.05,0.10,0.15,0.20,0.25,0.30,0.4,0.5,0.6},  
    % hide y axis
  ]

 \addplot [very thick,cyan!50!black] {gauss(x, 3, 2)};

\pgfmathsetmacro\valueC{gauss(3,3,2)}
	
\draw [gray] (axis cs:3,0) -- (axis cs:3,\valueC) ;
%\draw [gray] (axis cs:1,0) -- (axis cs:1,\valueA)   (axis cs:5,0) -- (axis cs:5,\valueA);
%\draw [gray] (axis cs:2,0) -- (axis cs:2,\valueB)    (axis cs:4,0) -- (axis cs:4,\valueB);
%\draw [yshift=1.4cm, latex-latex](axis cs:2, 0) -- node [fill=white] {$0.683$} (axis cs:4, 0);
%\draw [yshift=0.3cm, latex-latex](axis cs:1, 0) -- node [fill=white] {$0.954$} (axis cs:5, 0);


\end{axis}



\end{tikzpicture}

\end{document}
